% !TEX TS-program = xelatex
% !TEX encoding = UTF-8 Unicode
% !Mode:: "TeX:UTF-8"

\documentclass{resume}
\usepackage{zh_CN-Adobefonts_external} % Simplified Chinese Support using external fonts (./fonts/zh_CN-Adobe/)
% \usepackage{NotoSansSC_external}
% \usepackage{NotoSerifCJKsc_external}
% \usepackage{zh_CN-Adobefonts_internal} % Simplified Chinese Support using system fonts
\usepackage{linespacing_fix} % disable extra space before next section
\usepackage{cite}

\begin{document}
\pagenumbering{gobble} % suppress displaying page number

\name{张乐乐}

\basicInfo{
  \email{z150060772@126.com} \textperiodcentered\ 
  \phone{(+86) 18810461285} \textperiodcentered 男 \textperiodcentered 1989-02-18} 
%  \linkedin[billryan8]{https://www.linkedin.com/in/billryan8}}
 
\section{\faGraduationCap\  教育背景}
\datedsubsection{\textbf{中国科学院计算技术研究所}, 北京}{2011 -- 2014}
\textit{硕士}\ 计算机体系结构
\datedsubsection{\textbf{山东师范大学}, 山东, 济南}{2007 -- 2011}
\textit{学士}\ 计算机科学与技术

\section{\faWrench\  工作经历}
\datedsubsection{\textbf{VMware}, 北京}{2014 -- 至今}
\textit{高级软件工程师(Senior MTS)}\  NSBU网络platform团队

\section{\faCogs\ IT 技能}
% increase linespacing [parsep=0.5ex]
\begin{itemize}[parsep=0.5ex]
  \item 编程语言:C++,C,python。
  \item 虚拟网络管理/控制面。
  \item 高质量商业产品开发,测试,运维流程。
  \item Linux平台软件开发。
  \item 计算机网络基础知识。
  \item Linux网络。
  \item linux内核。
  \item X86虚拟化技术。
  \item 英语CET6,可熟练阅读英文文献。
\end{itemize}

\section{\faUsers\ 项目经历}
\datedsubsection{\textbf{VMware} 北京}{2020年3月 -- 至今}
\role{C++, python}{技术leader}
网络配置与vsphere生命周期管理器集成
\begin{itemize}
  \item 生命周期管理器是VMware下一代的server配置管理技术。利用规范化的schema和数据库存储server的所有配置,从而实现利用配置文档方便的管理整个集群甚至整个数据中心。
  \item 分析ESXi的所有网络配置持久化和恢复的行为,设计网络配置的schema。
  \item 带领团队重构底层库,实现vswitch,netstack,vmknic,pnic,firewall配置往数据库迁移。网络配置的持久化和恢复与以前相比变得规范化和统一。
  \item 带领团队实现基于schema的网络配置合法性检查模块,配置应用模块。从而网络配置可以灵活快速的推送到整个集群并生效。
\end{itemize}

\datedsubsection{\textbf{VMware} 北京}{2019年10月 -- 2020年3月}
\role{C++, python}{主要开发者}
\begin{onehalfspacing}
NSX-T NVDS到CVDS升级
\begin{itemize}
  \item 随着NSXT和vsphere的升级,客户需要把NSXT虚拟网络的data path由NVDS切换到CVDS。本项目为客户提供day0和day2的解决方案。
  \item 与NSXT团队共同设计整体的data path切换流程, 保证不同产品间在升级过程中相互协作。保证不同产品不同版本的兼容性。
  \item 在day0切换过程中编写反映升级阶段的vsphere API,用来与NSXT升级协作。
  \item 在day2切换过程中,编写NSXT上的命令行工具,包括precheck, 新的网络拓扑生成和迁移的命令。
  \item 功能已在vsphere7.0u1和NSXT Grindcore发布。
\end{itemize}
\end{onehalfspacing}

\datedsubsection{\textbf{VMware} 北京}{2019年8月 -- 2019年9月}
\role{C++}{设计,开发}
\begin{onehalfspacing}
VMware vsphere分布式交换机与NSXT集成后的autodeploy
\begin{itemize}
  \item autodeploy是指全新的机器随着上电可以自动安装和获取配置。
  \item 与NSXT团队讨论并设计方案, 编写设计文档。
  \item 负责vsphere端hostprofile代码的开发。
\end{itemize}
\end{onehalfspacing}

% Reference Test
%\datedsubsection{\textbf{Paper Title\cite{zaharia2012resilient}}}{May. 2015}
%An xxx optimized for xxx\cite{verma2015large}
%\begin{itemize}
%  \item main contribution
%\end{itemize}

\section{\faHeartO\ 获奖情况}
\datedline{\textit{第一名}, xxx 比赛}{2013 年6 月}
\datedline{其他奖项}{2015}

\section{\faInfo\ 其他}
% increase linespacing [parsep=0.5ex]
\begin{itemize}[parsep=0.5ex]
  \item 技术博客: http://blog.yours.me
  \item GitHub: https://github.com/username
  \item 语言: 英语 - 熟练(TOEFL xxx)
\end{itemize}

%% Reference
%\newpage
%\bibliographystyle{IEEETran}
%\bibliography{mycite}
\end{document}
